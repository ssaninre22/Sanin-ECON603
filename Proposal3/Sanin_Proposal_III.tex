%Input preamble
%Style
\documentclass[12pt]{article}
\usepackage[top=1in, bottom=1in, left=1in, right=1in]{geometry}
\parindent 22pt
\usepackage{fancyhdr}

%Packages
\usepackage{adjustbox}
\usepackage{amsmath}
\usepackage{amsfonts}
\usepackage{amssymb}
\usepackage[english]{babel}
\usepackage{bm}
\usepackage[table]{xcolor}
\usepackage{tabu}
\usepackage{color,soul}
\usepackage[utf8x]{inputenc}
\usepackage{makecell}
\usepackage{longtable}
\usepackage{multirow}
\usepackage[normalem]{ulem}
\usepackage{etoolbox}
\usepackage{graphicx}
\usepackage{tabularx}
\usepackage{ragged2e}
\usepackage{booktabs}
\usepackage{caption}
\usepackage{fixltx2e}
\usepackage[para, flushleft]{threeparttablex}
\usepackage[capposition=top]{floatrow}
\usepackage{subcaption}
\usepackage{pdfpages}
\usepackage{pdflscape}
\usepackage[sort&compress]{natbib}
\usepackage{bibunits}
\usepackage[colorlinks=true,linkcolor=darkgray,citecolor=darkgray,urlcolor=darkgray,anchorcolor=darkgray]{hyperref}
\usepackage{marvosym}
\usepackage{makeidx}
\usepackage{setspace}
\usepackage{enumerate}
\usepackage{rotating}
\usepackage{epstopdf}
\usepackage[titletoc]{appendix}
\usepackage{framed}
\usepackage{comment}
\usepackage{xr}
\usepackage{titlesec}
\usepackage{footnote}
\usepackage{longtable}
\newlength{\tablewidth}
\setlength{\tablewidth}{9.3in}
\usepackage[bottom]{footmisc}
\usepackage{stackengine}
\newcommand\barbelow[1]{\stackunder[1.2pt]{$#1$}{\rule{1ex}{.085ex}}}
\usepackage{titletoc}
\usepackage{accents}
\usepackage{arydshln }
\usepackage{titletoc}
\titlespacing{\section}{.2pt}{1ex}{1ex}
\setcounter{section}{0}
\renewcommand{\thesection}{\arabic{section}}


\makeatletter
\pretocmd\start@align
{%
  \let\everycr\CT@everycr
  \CT@start
}{}{}
\apptocmd{\endalign}{\CT@end}{}{}
\makeatother
%Watermark
\usepackage[printwatermark]{xwatermark}
\usepackage{lipsum}
\definecolor{lightgray}{RGB}{220,220,220}
\definecolor{dimgray}{RGB}{105,105,105}

%\newwatermark[allpages,color=lightgray,angle=45,scale=3,xpos=0,ypos=0]{Preliminary Draft}

%Further subsection level
\usepackage{titlesec}
\titleformat{\paragraph}
{\normalfont\normalsize\bfseries}{\theparagraph}{1em}{}
\titlespacing*{\paragraph}
{0pt}{3.25ex plus 1ex minus .2ex}{1.5ex plus .2ex}

\titleformat{\subparagraph}
{\normalfont\normalsize\bfseries}{\thesubparagraph}{1em}{}
\titlespacing*{\subparagraph}
{0pt}{3.25ex plus 1ex minus .2ex}{1.5ex plus .2ex}

%Functions
\DeclareMathOperator{\cov}{Cov}
\DeclareMathOperator{\sign}{sgn}
\DeclareMathOperator{\var}{Var}
\DeclareMathOperator{\plim}{plim}
\DeclareMathOperator*{\argmin}{arg\,min}
\DeclareMathOperator*{\argmax}{arg\,max}

%Math Environments
\usepackage{amsthm}
\newtheoremstyle{mytheoremstyle} % name
    {\topsep}                    % Space above
    {\topsep}                    % Space below
    {\color{black}}                   % Body font
    {}                           % Indent amount
    {\itshape \color{dimgray}}                   % Theorem head font
    {.}                          % Punctuation after theorem head
    {.5em}                       % Space after theorem head
    {}  % Theorem head spec (can be left empty, meaning ?normal?)

\theoremstyle{mytheoremstyle}
\newtheorem{assumption}{Assumption}
\renewcommand\theassumption{\arabic{assumption}}

\theoremstyle{mytheoremstyle}
\newtheorem{assumptiona}{Assumption}
\renewcommand\theassumptiona{\arabic{assumptiona}a}

\newtheorem{assumptionb}{Assumption}
\renewcommand\theassumptionb{\arabic{assumptionb}b}

\newtheorem{assumptionc}{Assumption}
\renewcommand\theassumptionc{\arabic{assumptionc}c}

\theoremstyle{mytheoremstyle}
\newtheorem{lemma}{Lemma}

\theoremstyle{mytheoremstyle}
\newtheorem{proposition}{Proposition}

\theoremstyle{mytheoremstyle}
\newtheorem{corollary}{Corollary}

%Commands
\newcommand\independent{\protect\mathpalette{\protect\independenT}{\perp}}
\def\independenT#1#2{\mathrel{\rlap{$#1#2$}\mkern2mu{#1#2}}}
\newcommand{\overbar}[1]{\mkern 1.5mu\overline{\mkern-1.5mu#1\mkern-1.5mu}\mkern 1.5mu}
\newcommand{\equald}{\ensuremath{\overset{d}{=}}}
\captionsetup[table]{skip=10pt}
%\makeindex

%Table, Figure, and Section Styles
\captionsetup[figure]{labelfont={bf},name={Figure},labelsep=period}
\renewcommand{\thefigure}{\arabic{figure}}
\captionsetup[table]{labelfont={bf},name={Table},labelsep=period}
\renewcommand{\thetable}{\arabic{table}}
\titleformat{\section}{\centering \normalsize \bf}{\thesection.}{0em}{}%\titlespacing*{\subsection}{0pt}{0\baselineskip}{0\baselineskip}
\renewcommand{\thesection}{\arabic{section}}

\titleformat{\subsection}{\flushleft \normalsize \bf}{\thesubsection}{0em}{}
\renewcommand{\thesubsection}{\arabic{section}.\arabic{subsection}}

%No indent
\setlength\parindent{24pt}
\setlength{\parskip}{5pt}

%Logo
%\AddToShipoutPictureBG{%
%  \AtPageUpperLeft{\raisebox{-\height}{\includegraphics[width=1.5cm]{uchicago.png}}}
%}

\newcolumntype{L}[1]{>{\raggedright\let\newline\\\arraybackslash\hspace{0pt}}m{#1}}
\newcolumntype{C}[1]{>{\centering\let\newline\\\arraybackslash\hspace{0pt}}m{#1}}
\newcolumntype{R}[1]{>{\raggedleft\let\newline\\\arraybackslash\hspace{0pt}}m{#1}} 

\newcommand{\mr}{\multirow}
\newcommand{\mc}{\multicolumn}

%\newcommand{\comment}[1]{}

% My New Commands






\begin{document}

\title{\Large \textbf{ECON 603 - Research Proposal III}}

\author{Sebasti\'an San\'in-Restrepo\thanks{Email: sr.sebastian@tamu.edu.}} 
\date{\today}

\maketitle
\thispagestyle{empty} 
\doublespacing
\thispagestyle{empty} 

\vspace{-10mm}
\pagenumbering{arabic}

\doublespacing

\section{ Research question}

\noindent \textbf{Question.} Did the cash assistance delivered during the pandemic relax liquidity constraints and increase entry into entrepreneurship especially among workers who had just lost a job?


\medskip
\noindent
\textbf{Context}. Beginning in 2020 the United States saw a historic and persistent rise in new business applications, with a notable increase in filings likely to become employers. At the same time, households received large cash transfers through the Economic Impact Payment. Levine and Rubinstein develop a three-sector Roy model in which the choice to become an incorporated entrepreneur requires both ability and collateral, so liquidity binds on that margin; empirically, they show incorporated entrepreneurs are positively selected while other self-employment behaves differently. \footnote{Ross Levine and Yona Rubinstein, 2019. \textit{Selection into Entrepreneurship and Self-Employment}}. 

\noindent \textbf{Importance and contribution.} This study asks whether pandemic cash injections nudged displaced workers toward incorporated entrepreneurship. We deliver a clear test of the liquidity channel by focusing on a single causal quantity, the interaction of recent job loss and local liquidity, and by linking CPS individual transitions to the contemporaneous surge in business applications (BFS). Thus, if liquidity eased binding constraints, entry should rise primarily into \textit{incorporated} self-employment, relative to \textit{unincorporated}, and most clearly among recent job losers. Pinning down this mechanism helps explain the post-2020 start-up boom and guides the design of emergency transfers and small-business policy in future downturns.

\section{ Economic framework and empirical design}

\noindent \textbf{Economic environment.} Workers choose among paid jobs, incorporated self-employment, and unincorporated self-employment. Liquidity binds on the incorporated margin because start-ups need upfront capital; job loss lowers the opportunity cost of entry. The pandemic added cash transfers and employment swings shocks. The model predicts cash should raise entry into incorporated self-employment, especially for workers recently laid off.


\noindent\textbf{Empirical design.} We test whether liquidity raises entrepreneurial entry among job-losers using two complementary models. First, with CPS data, we estimate a transition model for individuals not self-employed in $t\!-\!1$:
\begin{equation*}
	\Pr(E^{\mathrm{inc}}_{ist}=1)=\alpha_s+\tau_t+\beta\big(\text{JobLoss}_{it}\times \text{Liquidity}_{st}\big)+X_{ist}\gamma+\varepsilon_{ist}
\end{equation*}
where $E^{\mathrm{inc}}_{ist}$ is entry into \emph{incorporated} self-employment; $\beta>0$ implies cash waves boost entry for job-losers. Re-estimating with $E^{\mathrm{uninc}}_{ist}$ provides a placebo/contrast. Second, with BFS state series, we run an event-time design around payment dates:
\begin{equation*}
	\log Y_{st}=\alpha_s+\lambda_t+\sum_{k\neq -1}\delta_k\,\mathbf{1}\{t-t_0=k\}\times \text{Exposure}_s+u_{st}
\end{equation*}
with $Y_{st}$ the number of business applications likely to become employers. Identification rests on parallel trends for job-losers across states absent cash waves and the exogenous national timing of EIP; we include state and month fixed effects, check pre-trends and placebo dates.

\section{ Data}

\noindent Data are public sources. Current Population Survey monthly microdata (state, class of worker, employment/separations) to build month-to-month transitions; Census Business Formation Statistics for state monthly series; IRS information on for EIP timing and exposure.



\singlespacing
\bibliographystyle{chicago}
%\bibliography{humancapital,ihdp,productionfunctions,econometrics}

\pagebreak
\renewcommand*{\thepage}{A.\arabic{page}}
\setcounter{page}{0}
\setcounter{equation}{0}
\renewcommand{\theequation}{A.\arabic{equation}}
\setcounter{section}{0}
\renewcommand{\thesection}{A.\arabic{section}}
\renewcommand{\thefigure}{A.\arabic{figure}}
\setcounter{figure}{0}
\renewcommand{\thetable}{A.\arabic{table}}
\setcounter{table}{0}
\thispagestyle{empty}


\end{document}