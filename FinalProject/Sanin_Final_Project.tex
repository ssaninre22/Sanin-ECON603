%Input preamble
\input{preamble}
\begin{document}

\title{\Large \textbf{ECON 603 - Research Proposal I}}

\author{Sebasti\'an San\'in-Restrepo\thanks{Department of Economics, Texas A\&M University. Address: 2935 Research Parkway, College Station, TX 77845, USA. Email: sr.sebastian@tamu.edu.}} 
\date{\today}

\maketitle
\thispagestyle{empty} 
\doublespacing
\thispagestyle{empty} 

\vspace{0mm}
\pagenumbering{arabic}

\doublespacing

\section*{IDEAS}
\begin{itemize}
	\item Commodity-price bank exposure: Do oil/coal price shocks transmit to banks via borrower risk in exposed regions?, Shift-share (regional exposure × global price); bank-region DiD on NPLs/provisions., 


\end{itemize}
\section{ Research question} \label{section:rq} 

\begin{itemize}
	\item Propose a research question that is interesting to you
	\item If you are already familiar with related papers or the context, provide a few
	sentences discussing the possible contribution of the research question.
	\item Does the possible contribution depend on the exact results? For example, do the
	results need to go in a different direction than the current literature for the paper
	to be a contribution?
\end{itemize}

\noindent \textbf{Home Visits.} Professional home visitors visited the households of the treated children, training the parents in problem-solving and parenting skills, following the curriculum \textit{Partners for Learning}. Professionals demonstrated, practiced, and discussed the curriculum with parents to train them as partners in their children's learning. Home visits occurred weekly during the first year of treatment and twice a month in the second and third years

\section{ Economic framework and empirical design} \label{section:efed} 

\begin{itemize}
	\item Consider the questions in Varian (2016): “Who are the people making the choices?
	What are the constraints that they face? How do they interact?” Is some observed
	behavior inconsistent with what one would expect? If there is a policy or shock
	in your research question, which elements of the model may be affected?
	\item What reduced-form results would be useful to show? What possible empirical
	design would allow you to show those results? What do you need for that design
	(e.g., a time-varying instrument, a spatial discontinuity)?
	\item What assumptions are needed? To start, can you further simplify the setup to
	reduce these assumptions?
\end{itemize}


\section{ Data} \label{section:data} 

\begin{itemize}
	\item Data source and access: Are the data public-access? If not, what steps and costs
	are involved to access the data? Do you have any leads or contacts?
	\item Units and time: What is a unit in the data? Do the data span multiple time
	periods? If so, are the data cross-sectional or longitudinal? What is the sampling
	or population of the data? How do these aspects support the economic framework
	and empirical design?
	\item Variables: What variables do the data contain that would be useful for the pro-
	posed analysis? What variables are missing that you would need to look for in
	other sources?
	\item If you have no idea about possible data sources for your research questions, write
	out what you would need in the ideal dataset. This includes if you think an
	experiment would be needed.
\end{itemize}




\singlespacing
\bibliographystyle{chicago}
%\bibliography{humancapital,ihdp,productionfunctions,econometrics}

\pagebreak
\renewcommand*{\thepage}{A.\arabic{page}}
\setcounter{page}{0}
\setcounter{equation}{0}
\renewcommand{\theequation}{A.\arabic{equation}}
\setcounter{section}{0}
\renewcommand{\thesection}{A.\arabic{section}}
\renewcommand{\thefigure}{A.\arabic{figure}}
\setcounter{figure}{0}
\renewcommand{\thetable}{A.\arabic{table}}
\setcounter{table}{0}
\thispagestyle{empty}


\end{document}