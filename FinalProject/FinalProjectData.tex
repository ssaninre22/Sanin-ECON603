% Project: `` Anchor institutions as macroeconomic stabilizers"
% Section: Data

\newpage
\section{ Data}

\noindent The empirical analysis combines several public and restricted-access datasets that provide information on university activity, local labor markets, housing, and business dynamics for U.S. counties and metropolitan areas from 1990 \footnote{Note 1: I'm still exploring the availability of data. I would like to capture the higher number of recessions possible. However, the earliest data for the unemployment by county/city is 1990. Thus, I would have to restrict my dataset to this point. Also, I need to see quality and availability of other series by country.} to 2023. The core measure of university dependence is constructed using data from the Integrated Postsecondary Education Data System (IPEDS) administered by the National Center for Education Statistics (NCES, 2024). IPEDS reports annual enrollment, employment, and payroll data for every degree-granting institution in the United States, allowing the construction of county-level indicators such as full-time-equivalent (FTE) students per capita, the share of total local employment accounted by the university, and the share of total local payroll accounted for by higher education (NAICS 61). These variables capture both the scale of universities relative to the local economy and their role as employment anchors.

\noindent Local labor-market outcomes come primarily from the Bureau of Labor Statistics’ Quarterly Census of Employment and Wages (QCEW), which provides monthly establishment-level data aggregated to the county level by industry. I focus on employment and payroll in total private employment and in non-tradable sectors most exposed to local demand, specifically, retail trade (NAICS 44–45), accommodation and food services (NAICS 72), arts and entertainment (NAICS 71), and real estate (NAICS 53). County unemployment rates are obtained from the Local Area Unemployment Statistics (LAUS), which provides information at the county or even city level\footnote{Note 2: I'm still exploring the data to look for the possibility to explore city-level estimation which will be more precise because of the University locations}. Demographic and structural controls, including population, educational attainment, and income, are taken from the American Community Survey (ACS) and decennial censuses compiled by the U.S. Census Bureau. Housing-market outcomes are drawn from Zillow’s ZORI rent index and FHFA house price series, which together provide consistent measures of rents and property values at the county or metropolitan level \footnote{Note 3: In my current theoretical model, I'm not having housing sector, although it was in my research proposal. I'm still working on the theoretical background, so maybe I will use this data later or not.}.

\noindent Recession periods are identified following the National Bureau of Economic Research (NBER) Business Cycle Dating Committee, which provides official monthly peaks and troughs of U.S. economic activity. These dates are used to define recession years for the 2001, 2008–09, and 2020 downturns. The preferred frequency of estimation depends on the dynamics of adjustment: monthly data would ideally capture the onset and recovery of local labor markets—particularly the timing of unemployment and payroll responses, whereas annual aggregation smooths short-run movements but better reflects cumulative effects. In practice, I will aggregate monthly data to quarterly frequency to match the unemployment and employment-by-industry datasets to begin with, to ensure temporal consistency while retaining enough frequency to track the differential speed of contraction and recovery across regions. Potential data challenges include incomplete reporting by colleges in IPEDS, gaps in early years for housing indicators, and the need to match institutions precisely to counties or CBSAs when universities straddle multiple jurisdictions. Finally, college towns will be defined based on their exposure to universities, typically those where at least 20–25 percent of total local employment is generated by higher-education institutions, although the empirical analysis will also explore alternative thresholds and continuous measures of university dependence to assess the robustness of results.








