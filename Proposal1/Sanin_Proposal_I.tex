%Input preamble
\input{preamble}

\begin{document}

\title{\Large \textbf{ECON 603 - Research Proposal I}}

\author{Sebasti\'an San\'in-Restrepo\thanks{Email: sr.sebastian@tamu.edu.}} 
\date{\today}

\maketitle
\thispagestyle{empty} 
\doublespacing
\thispagestyle{empty} 

\vspace{-10mm}
\pagenumbering{arabic}

\doublespacing

\section{ Research question} \label{section:rq} 

\noindent \textbf{Question.} What role does banks' \textit{internal liquidity reallocation}\footnote{It refers to within-bank shifting of funding capacity as well as lending space, across municipalities, coordinated by the head office, that injects liquidity into some locations and withdraws it from others} play in shaping banks' \textit{risk-taking behavior} in Colombia? Evidence from commodity price shocks.


\medskip
\noindent
\textbf{Context and why it matters.} Colombia’s commodity geography (oil, coal, coffee) induces uneven local income cycles. Positive price shocks raise households and firms cash flows and expand the stock of low-cost deposits; negative shocks reverse them. Banks may respond by expanding safer, collateralized credit or by \textit{reaching for yield} in consumer or microcredit lending when funding is abundant. Internal capital markets let head offices \textit{reallocate liquidity} across municipalities, potentially amplifying or muting local risk-taking. Because macroprudential tools act on these margins, identifying the reallocation channel is policy-relevant.


\medskip
\noindent
\textbf{Contribution.} I estimate \textit{risk-taking elasticities} to \textit{internal liquidity reallocation}. Leveraging exogenous commodity shocks and bank-municipality panels, I use predicted within-bank inflows/outflows to causally shift the risky loan share while holding local demand and deposit supply fixed, then test whether boom-time tilts translate into losses. The result is a set of geography- and bank-specific elasticities for macroprudential analysis.


\section{ Economic framework and empirical design} \label{section:efed} 

\noindent \textbf{Economic environment.} Banks choose quantities, prices, and loan composition under liquidity (deposits) constraints;  while households and firms in municipalities demand credit and supply deposits. Exogenous commodity price movements generate income shocks in producing areas. These shocks shift local deposit supply and borrower default risk, and trigger within-bank reallocation of resources across places.

\noindent \textbf{Reduced-form.} I will show that when a bank-branch faces liquidity slack or stress coming from commodity-driven income shocks, its head office reallocates funds and the local risky mix (consumer and microcredit loans share) shifts and how much of that shift is due to \textit{local funding slack} versus \textit{internal reallocation}.
 
\noindent \textbf{Design and requirements.} The design combines \textit{instrumental variables} and \textit{difference-in-differences}. For IV a Bartik-style instrument for municipal income shocks to identify exogenous variation and a leave-one-out bank-wide shock to capture internal liquidity reallocation that is orthogonal to own shock. In DiD, a two-way fixed effects estimation deliver within-municipality and within-bank comparisons.

\noindent \textbf{Assumptions and simplifications.} I assume that global prices are exogenous, pre-period weights (exposure, funding reliance) are stable shifters. To start conservatively I could focus on one commodity (i.e., oil) and begin with the contemporaneous composition result, then add dynamics and hazards.


\section{ Data} \label{section:data} 

\noindent The core dataset is a bank$\times$municipality$\times$quarter panel with loan and deposits stocks by type and other financial variables (Financial Superintendence). Exogenous shocks use global commodity prices (IMF) combined with pre-period municipal exposures (Municipal panel, Universidad de los Andes). Other control variables (DANE) are also available. All data considered for this project is publicly available.


\singlespacing
\bibliographystyle{chicago}
%\bibliography{humancapital,ihdp,productionfunctions,econometrics}

\pagebreak
\renewcommand*{\thepage}{A.\arabic{page}}
\setcounter{page}{0}
\setcounter{equation}{0}
\renewcommand{\theequation}{A.\arabic{equation}}
\setcounter{section}{0}
\renewcommand{\thesection}{A.\arabic{section}}
\renewcommand{\thefigure}{A.\arabic{figure}}
\setcounter{figure}{0}
\renewcommand{\thetable}{A.\arabic{table}}
\setcounter{table}{0}
\thispagestyle{empty}


\end{document}