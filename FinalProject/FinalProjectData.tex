% Project: `` Anchor institutions as macroeconomic stabilizers"
% Section: Data

\newpage
\section{ Data}

\noindent The empirical analysis draws on several public and restricted-access datasets that provide information on university activity, local labor markets, and housing for U.S. counties and metropolitan areas from 1990 to 2023.\footnote{I am still exploring data availability for the early 1990s. The earliest consistent county-level unemployment data are available from 1990, which may restrict the final sample. The quality and coverage of other series will also determine the lower bound of the time frame.} The main measure of university dependence comes from the Integrated Postsecondary Education Data System (IPEDS) administered by the National Center for Education Statistics (NCES, 2024). IPEDS reports annual data on enrollment, employment, and payroll for all degree-granting institutions in the United States. Using these data, I construct county-level indicators such as full-time-equivalent (FTE) students per capita, the share of total local employment generated by higher education, and the share of total local payroll in education (NAICS 61). These variables capture the scale of universities relative to their local economies and their importance as employment anchors.

\noindent I obtain local labor market outcomes primarily from the Bureau of Labor Statistics’ Quarterly Census of Employment and Wages (QCEW), which provides monthly establishment-level data aggregated to the county level by industry. I focus on total employment and payroll, as well as on non-tradable sectors most sensitive to local demand: retail trade (NAICS 44–45), accommodation and food services (NAICS 72), arts and entertainment (NAICS 71), and real estate (NAICS 53). I collect county unemployment rates from the Local Area Unemployment Statistics (LAUS), which provides consistent information at the county and, in some cases, city level.\footnote{I am also exploring the feasibility of city-level estimation, which could improve precision by aligning more closely with university locations.} Demographic and structural controls, including population, income, and educational attainment, come from the American Community Survey (ACS) and the decennial censuses compiled by the U.S. Census Bureau. I also compile housing market outcomes from Zillow’s ZORI rent index and the Federal Housing Finance Agency (FHFA) house price series, which together provide consistent measures of rents and property values at the county or metropolitan level.\footnote{The current theoretical model does not explicitly include a housing sector, but these data may become relevant as the theoretical framework evolves.}

\noindent I identify recession periods using the official monthly peaks and troughs published by the National Bureau of Economic Research (NBER) Business Cycle Dating Committee. These dates define recession months for the 2001, 2008–2009, and 2020 downturns. The choice of data frequency depends on the dynamics of adjustment I aim to capture: monthly data allow for a detailed view of the onset and recovery of local labor markets, while annual data summarize cumulative effects. In practice, I aggregate monthly data to quarterly frequency to align unemployment and industry-level employment data, ensuring temporal consistency while retaining enough variation to track differences in contraction and recovery across regions. The main data challenges involve incomplete reporting by small colleges in IPEDS, limited housing data for earlier years, and the need to match universities precisely to counties or core-based statistical areas when institutions span multiple jurisdictions. I define college towns as counties where higher education accounts for at least 20 to 25 percent of total local employment, but I also estimate models using alternative thresholds and a continuous measure of university exposure to assess robustness.


%\textbf{AZ Comments:} This is a wonderful draft of the Data section. I suggest editing it to reduce the use of passive voice and only use present tense (even if you have not yet done what you are describing). This helps make it more aligned with academic writing that is published. I am surprised you cannot get data on unemployment by count/city from before 1990. I would expect Census or BLS to have some data products for that. Mostly addressed. Still looking for data prior to 1990.





