% Project: `` Anchor institutions as macroeconomic stabilizers"
% Section: Literature Review

\newpage
\section{ Literature Review}

\noindent A growing literature investigates how universities function as anchor institutions that shape local economic resilience, employment, and spatial development. Classical urban and regional studies view universities as immobile organizations that generate backward and forward linkages, local spending multipliers, and long-term knowledge spillovers \citep{Saxenian1996RegionalAdvantage,felsensteinUniversityMetropolitanArena1996}. Recent measurement advances make this idea operational. The Philadelphia Fed's Anchor Economy Initiative documents which U.S. communities are most reliant on universities for employment and income, highlighting college towns where a single campus accounts for a large share of local GDP and jobs \citep{Harker2024AnchorReliance}. This reliance is worthy; it implies that university scale and in-person presence can transmit shocks to non-tradable sectors (housing, retail, restaurants) and, conversely, dampen downturns when university activity expands or remains stable. Overall, the literature distinguishes between the long-run structural impacts of universities on local development and their short-run cyclical role as stabilizers of regional economies.

\noindent \textbf{Universities and long-run local development.} In the long run, universities raise the local stock of skills and foster innovation, but the magnitude of these effects depends on graduate retention, industrial structure, and local absorption capacity. Evidence from the United States and Europe shows that while university openings or expansions attract students and increase educational attainment, they yield heterogeneous impacts on wages and employment, often concentrated in dynamic or high-tech regions \citep{berlingieriCollegeOpeningsLocal2022,ferhatImpactUniversityOpenings,amendolaDoesGraduateHuman2020,abelCollegesUniversitiesIncrease2012}. Other work highlights that research universities, through academic R\&D and knowledge spillovers, can modify the local occupational composition even when direct employment effects are modest \citep{beesonEffectsCollegesUniversities,harrisUniversitiesAnchorInstitutions2016}. These studies show that universities' long-run influence operates through both human capital accumulation and persistent changes in regional specialization.

\noindent \textbf{The role of universities as cyclical stabilizers.} A second line of work focuses on universities' short-run stabilizing role in local economies. Empirical studies in labor and education economics consistently find that schooling is countercyclical: when labor market prospects weaken, the opportunity cost of education declines and enrollments rise, especially among teens and prime-aged adults who view schooling as an investment in employability \citep{dellasBusinessCyclesSchooling2003,bettsSafePortStorm1995}. Because university payrolls and student spending inject relatively stable demand into local non-tradable sectors, this countercyclicality can act as a built-in stabilizer for college-town economies. In regional science, this mechanism aligns with the ``resistance" dimension of regional resilience, defined as the capacity of a region to experience smaller initial declines in output or employment after a shock \citep{martinNotionRegionalEconomic2015,martinRegionalEconomicResilience2012}. 

\noindent Consistent with this mechanism, several recent papers provide direct empirical evidence on stabilization. \citet{howardUniversitiesImproveLocal2024} show that historical variation in the location of regional public universities explains greater employment resilience during sectoral downturns, driven largely by consumption within local non-tradables, with faculty and student populations channeling external income into local spending. However, these effects are heterogeneous. Research-intensive campuses with substantial R\&D, medical centers, or graduate programs generate stronger and more durable local demand and knowledge spillovers, whereas smaller or teaching-focused colleges in thinner markets provide more limited stabilization \citep{howardUniversitiesImproveLocal2024,beesonEffectsCollegesUniversities}. Evidence from the openings or expansions literature reinforces this heterogeneity, finding that instructional growth reliably increases student inflows and local high-skill shares, but broader labor-market multipliers depend on sectoral absorptive capacity and the spatial frictions that shape graduate retention \citep{berlingieriCollegeOpeningsLocal2022,ferhatImpactUniversityOpenings,amendolaDoesGraduateHuman2020}.


\noindent Conversely, the same dependence that dampens typical downturns may become a source of fragility when the anchor pulls back. The regional-resilience framework emphasizes that the nature of the shock matters. When disturbances directly suppress the anchor's local footprint, such as temporary campus suspensions or institutional closures, the usual insurance channel weakens, and exposure in services and housing becomes a liability \citep{martinNotionRegionalEconomic2015}. Case-based and emerging empirical work on university-dependent places documents sharp short-run contractions in restaurant and retail activity, small-business survival, and rental markets when students abruptly depart, consistent with a sudden-stop mechanism in non-tradables \citep{howardUniversitiesImproveLocal2024,Harker2024AnchorReliance}. In this sense, university towns exhibit a stability–fragility tradeoff, they are buffered when enrollment rises countercyclically or payrolls persist, yet vulnerable when the campus itself cannot operate in person or when institutions shutter.


\noindent \textbf{Closest work to this study.} \citet{jumpResearchUniversitiesRecession} examine whether counties hosting state flagship universities experienced greater resilience across the 2001, 2008–09, and 2020 U.S. recessions, they employ synthetic difference-in-differences methods to show no significant effect during the 2001 downturn, a strong cushioning effect in the Great Recession, and a negative effect in 2020, when most campuses suspended in person activities. These findings reinforce the idea that stable local consumption from university populations supports resilience in ordinary recessions but disappears when campus presence collapses. This paper is thus closest in spirit to the present study, which extends their analysis beyond flagship universities to a broader range of college towns and examines heterogeneity by university dependence, housing-market tightness, and service-sector exposure.

\noindent Taken together, the literature points to a clear stability–fragility tradeoff in university-dependent economies. Universities act as anchors that can buffer local downturns through countercyclical enrollment and steady payrolls, yet their immobility and spatial concentration also generate exposure when in-person activity declines or institutions close. The evidence to date, particularly from \citet{jumpResearchUniversitiesRecession}, provides a strong empirical precedent but remains limited to flagship universities and aggregate labor-market outcomes. The present project builds on and extends this work by analyzing a wider set of college towns, incorporating housing and service-sector dynamics, and testing how varying degrees of university dependence mediate both stabilizing and destabilizing effects. In doing so, it contributes to a deeper understanding of how anchor institutions shape regional economic resilience and the asymmetric transmission of shocks across space.



